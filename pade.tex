% Pacotes
\documentclass[12pt]{article}
\usepackage{adjustbox}
\usepackage[utf8]{inputenc}
\usepackage{amsmath}
\usepackage{sbc-template}
\usepackage{amsfonts}
\usepackage{amsmath}
\usepackage{graphicx,url}


\sloppy

\title{Aproximantes de Padé}

\author{Ricardo Henrique Brunetto\inst{1}}

\address{Departamento de Informática -- Universidade Estadual de Maringá (UEM)\\
	Maringá -- PR -- Brasil
	\email{ra94182@uem.br}
}

\begin{document}

	\maketitle

  \section{Introdução}

	Datado da década de 1890 e de autoria de Henri Padré,  o \textbf{Aproximante de Padé} é
	a melhor aproximação conhecida  de uma função através de uma função racional de determinada ordem.
	Nesta técnica, inclusive, o aproximante de uma série de potência é compatível com a série de potência
	da função que se está aproximando. Isso significa que os Aproximantes de Taylor são casos particulares dos Aproximantes
	de Padé.

	Em termos gerais, o resultado produzido pela aproximação de Padé se mostra mais preciso
	que o truncamento da série de Taylor da função, produzindo, inclusive, resultados em casos
	onde a série de Taylor da função não converge. Por esse motivo, é comum utilizar o método de
	Padé nos cálculos computacionais, ainda que como funções auxiliares.

	Além disso, os aproximantes de Padé não somente convergem mais rapidamente que a série de Taylos como,
	também, se estende a regiões muito além do raio de convergência da mesma. Os coeficientes dos aproximantes
	de Padé são baseados apenas nos coeficientes da Série de Taylor da função que se deseja aproximar.

	\section{Definição}
	Dada uma função $f$ e $m,n \in \mathbb{Z}$ com $m\geq0, n\geq1$, o \textit{Aproximante de Padé}
	de $f$ de ordem $[m/n]$ é a função racional $R$, denotada por $[m/n]_f(x)$, dada por:
	$$R(x) = \frac{\sum_{j=0}^{m}{a_jx^j}}{1 + \sum_{k=1}{n}{b_kx^k}} = \frac{a_0 + a_1x + \dots + a_mx^m}{1 + b_0 + \dots + b_nx^n}$$

	Percebe-se que, ao tomar as derivadas da função $f$ e as derivadas de $R$, ambas com $x=0$, tem-se que a $k$-ésima derivada
	de $f$ em $x=0$ é igual à $k$-ésima derivada de $R$ em $x=0$. Por consequência,
	$$f^{(m+n)} = R^{(m+n)}$$

	Em decorrência disso, os coeficiêntes da função racional que define o Aproximante de Padé são únicas. Ou seja,
	$a_i\,,i\in\{0,\dots,m\}$ e $b_j\,,j\in\{1,\dots,n\}$ só podem ser unicamente determinados. Por esse motivo, $b_0 = 1$,
	caso contrário o numerador de $R$ poderia ser multiplicado por constantes, que tirariam sua unicidade (normalização).

	\section{Cálculo Computacional}
	\label{sec:dev}
	Computacionalmente, os aproximantes de Padé podem ser calculados pelo Algoritmo Epsilon, desenvolvido por Peter Wynn,
	através de transformações em somas parciais. Dada a série de Taylor
	$$T_n(x) = c_0 + c_1x + \dots + c_nx^n$$
	onde $c_k = \frac{f^{(k)}(0)}{k!}$.

	Além disso, se $f$ for escrita como uma série de potências formalmente, pode-se aplicar o Algoritmo de Euclides para a divisão
	entre polinômios, tendo assim o maior divisor comum:
	$$R(x) = P(x)/Q(x) = T_{m+n}(x)\;mod\;x^{m+n+1}$$

	Assim, deve existir $K(x)$ tal que:
	$$P(x) = Q(x)\times T_{m+n}(x) + K(x)\times x^{m+n+1}$$

	Dessa forma, $f(x) - R(x)$ deve ter uma raiz $r$ com multiplicidade $m+n+1$. Para que isso ocorra, os coeficientes $x^k$ do numerador
	devem se anular:
	\begin{equation}\label{eq:max}
	\sum_{i=0}^{k}{(c_ib_{k-i} - a_k)} = 0\;k \in \{0, \dots, m+n\}
	\end{equation}

	Dessa forma, encontram-se os coeficientes para os aproximantes de Padé.

	Isso implica também que $f$ e o aproximante de Padé $R$ diferem apenas por termos da ordem $x^{m+n+1}$. Como
	os termos se tornam cada vez menores, o Teorema do Resto garante a precisão baseada na escolha de $m$ e $n$.

	\section{Exemplos}
	\paragraph{$f(x) = e^x$}
	Considerando tal função, desenvolver-se-á o Aproximante de Padé $f(x) = e^x$ considerando $m = 2$ e $n = 2$.
	Dessa forma, deseja-se:
	$$R(x) = \frac{a_0 + a_1x + a_2x^2}{1 + b_1x + b_2x^2}$$

	Além disso, sabe-se que a série de Taylor de $f(x)$ é tal que:
	$$f(x) = 1 + x + \frac{x^2}{2!} + \frac{x^3}{3!} + \dots$$

	Dessa forma, expandindo \ref{eq:max} desenvolvida na Seção \ref{sec:dev}:
	\begin{align}
		c_0 \times b_0 = a_0 & \implies a_0 = b_0 = 1\\
		c_0 \times b_1 + c_1 \times b_0 = a_1 & \implies b_1 + 1 = a_1\\
		c_0 \times b_2 + c_1 \times b_1 + c_2 \times b_0 = a_2 & \implies b_2 + b_1 + \frac{1}{2} = a_2\\
		c_1 \times b_2 + c_2 \times b_1 + c_3 \times b_0 = 0 & \implies b_2 + \frac{b_1}{2} + \frac{1}{6} = 0\\
		c_2 \times b_2 + c_3 \times b_1 + c_4 \times b_0 = 0 & \implies \frac{b_2}{6} + \frac{b_1}{6} + \frac{1}{24} = 0\\
	\end{align}
	O que gera um \textbf{Sistema de Equações Lineares}. Aplicando técnicas de resolução, tem-se:
	$$a_0 = b_0 = 1$$
	$$b_1 = \frac{-1}{2}$$
	$$b_2 = \frac{1}{12}$$
	$$a_1 = \frac{1}{2}$$
	$$a_2 = \frac{1}{12}$$
	Dessa maneira, tem-se $R(x)$ como sendo:
	$$f(x) \approx R(x) = \frac{1 + \frac{x}{2} + \frac{x^2}{12}}{1 - \frac{x}{2} + \frac{x^2}{12}} = \frac{12 + 6x + x^2}{12 - 6x + x^2} = \frac{(x+3)^2 + 3}{(x-3)^2 + 3}$$
	$$\iff$$
	$$e^x \approx \frac{(x+3)^2 + 3}{(x-3)^2 + 3}$$
	\bibliographystyle{sbc}
	\bibliography{references}
	\nocite{*}
\end{document}
